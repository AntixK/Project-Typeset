\documentclass[a4paper,11pt, leqno]{article}
\usepackage{enumitem}
\usepackage{geometry}
\usepackage{multicol}
\usepackage{mathtools}
\usepackage{setspace}
\usepackage{ragged2e}
\usepackage{amsfonts}
\usepackage{hyperref}
\usepackage[utf8]{inputenc}
\usepackage[english, dutch]{babel}
\usepackage{tikz}
\usepackage{color}
\usepackage{amsthm}
\usepackage[ruled]{algorithm}% http://ctan.org/pkg/algorithms
\usepackage{algpseudocode}
\usepackage{graphicx}
\usepackage[T1]{fontenc}
\usepackage[nodayofweek,level]{datetime}
\usepackage[section]{placeins} 

\setcounter{equation}{-1}
\title{Coxeter's Rabbit}
\author{Edsger W. Dijkstra}
\date{}
\begin{document}
\maketitle
 On p.13 of his "Introduction to Geometry", H.S.M Coxeter invites the reader to see (and to use spontaneously) that with $s = (a+b+c)/2$, $abc$ equals
\begin{equation}
s(s-b)(s-c) + s(s-c)(s-a) + s(s-a)(s-b) - (s-a)(s-b)(s-c)
\end{equation}
\underline{Proof}
\begin{align}
\begin{aligned}[t]
s(s-b)(s-c) + s(s-c)(s-a) &= s(s-c)(2s - a-b)  &\text{\{algebra\}}\\
&= s(s-c)c  &\text{\{definition of }s\}
\end{aligned}
\end{align}

\begin{align}
\begin{aligned}[t]
s(s-a)(s-b) - (s-a)(s-b)(s-c) &= (s-a)(s-b)c  &\text{\{algebra\}}
\end{aligned}
\end{align}
Because both expression $(1)$ and $(2)$ contain a factor $c$, so does $(0)$; for reasons of symmetry, $(0)$ also contains factors $a$ and $b$, i.e. is a multiple of $abc$. The c\"oefficient equals $1$ -as is trivially established with, say, $a,b,c := 2,2,2$- and thus $abc = (0)$ has been proved.

\begin{flushright}
(End of Proof)\\[4pt]

Nuenan, 14 April 2002
\end{flushright}

Prof. Dr. Edsger W. Dijkstra

Plantaanstraat 5

5671 AL NUENEN

The Netherlands

\end{document}